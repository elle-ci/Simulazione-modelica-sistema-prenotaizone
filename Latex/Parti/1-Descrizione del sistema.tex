\chapter*{Descrizione del sistema}
Con questo progetto, sviluppato mediante Modelica versione 1.7 affiancato a  Python 2.7 (utilizzato per verificare l’andamento del sistema su un numero elevato di iterazioni), si   vuole modellare ad alto livello un sistema di prenotazione delle aule di un’università.\\
Gli attori in gioco nel sistema sono: le \textbf{aule}, gli \textbf{studenti} che rappresentano gli utenti finali il cui interesse principale è quello di effettuare una prenotazione o eventualmente una cancellazione di una prenotazione, il \textbf{Gomp}, un sistema esterno che ha il compito di immagazzinare e gestire varie informazioni mettendo quindi in contatto gli elementi del sistema, infine \textbf{Prodigit} l’interfaccia del sistema di prenotazione con cui interagisce lo studente, in particolare permette  di prenotare un posto in aula (se il posto è disponibile e se l’aula è agibile) o di cancellare una prenotazione  se almeno un posto nell’aula è occupato.  Il sistema inoltre contiene anche dei monitor che hanno il compito di verificare  che i requisiti vengano rispettati.
\newline
\par Il focus principale è quello di gestire la comunicazione tra i vari membri del sistema preoccupandosi della concordanza tra i dati e delle tempistiche di comunicazione, cercando di rendere il sistema quanto più reale e coerente possibile. 

